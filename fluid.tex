\documentclass{jsarticle}
\usepackage[top=30truemm, bottom=30truemm, left=25truemm, right=25truemm]{geometry}
\usepackage{amssymb, amsmath}
\usepackage{bm}

\newcommand{\pfrac}[2]{\frac{\partial{}{#1}}{\partial{}{#2}}}

\begin{document}
\section{はじめに}
空気,水,ガス,ガソリン.これらに共通することはなんだろうか.日々の生活に欠かせないもの.確かに間違ってはいない.空気と水がなければ僕たちは行きていけないし,ガスがなければ料理ができず,ガソリンがなければ車に乗れない.しかしこれ以上にこれらに共通することがある.「流体」であるということだ.

「流れる体」と書いて「流体」だが,それは一体何を示すのだろうか.答えは「個体以外のもの全て」だ.実際にはもっと厳格な定義があるが,それはここでは触れないことにしよう.

本稿の目的は,流体を表現する様々な式に触れながら,流体運動を記述するナビエストークス方程式を導出することだ.この方程式は現在様々な分野で活躍する数値流体力学の基礎となる数式で,この方程式さえわかれば流体運動がわかると言っても過言ではない.では,流体を支配する世界を紐解いていこう.

\section{連続の式}
これから登場する様々な式の中で,流体に対して最も強い拘束条件を与えるのが連続の式である.この式は「流体は突然動き出さない」ということを意味する.早速この数式を見てみよう.
\begin{equation}
  \pfrac{\rho}{t}+\pfrac{\rho{}u}{x}+\pfrac{\rho{}v}{y}+\pfrac{\rho{}w}{z}=0
\end{equation}
これをベクトル演算子を使って書いてみると,
\begin{equation}
  \pfrac{\rho}{t}+\mathrm{div}(\rho\bm{V})
\end{equation}
となる.この数式は,密度変化と質量変化の総和は常にゼロに等しいことを意味する.ではどのようにこの数式が導出されるのかを次に述べる.
\subsection{連続の式の導出: 直交座標系の場合}
とても小さな立方体を考えよう.一辺の長さが$\mathrm{dx}$,$\mathrm{dy}$,$\mathrm{dz}$の立方体のそれぞれの面に流入,流出する流体の質量を考える.

まず$x$軸に垂直な面に$x$軸方向の流れが流入,流出するとき,
\begin{align}
  &流入:\rho{}u|_{x}dydzdt\label{xin}\\
  &流出:\rho{}u|_{x+dx}dydzdt\label{xout}
\end{align}
となり,式(\ref{xout})に関してテイラー展開,一次までを考えると,
\begin{equation}
  \left[\rho{}u+\pfrac{(\rho{}u)}{x}\right]dydzdt\label{xouttaylor}
\end{equation}
となり,ここで流入と流出の差,つまり式(\ref{xin})と式(\ref{xouttaylor})の差を取れば,
\begin{equation}
  -\pfrac{(\rho{}u)}{x}dydzdt
\end{equation}
を得る.これを$y$に垂直な面,$z$に垂直な面についてそれぞれ考えると,
\begin{align}
  &-\pfrac{(\rho{}u)}{x}dydzdt\\
  &-\pfrac{(\rho{}v)}{y}dydzdt\\
  &-\pfrac{(\rho{}w)}{z}dydzdt
\end{align}
がそれぞれ得られる

\end{document}
