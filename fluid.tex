\documentclass{jsarticle}
\usepackage[top=30truemm, bottom=30truemm, left=25truemm, right=25truemm]{geometry}
\usepackage{amssymb, amsmath}
\usepackage{bm}
\usepackage[dvipdfmx]{graphicx}

\newcommand{\pfrac}[2]{\frac{\partial{}{#1}}{\partial{}{#2}}}

\begin{document}
\section{はじめに}
空気,水,ガス,ガソリン.これらに共通することはなんだろうか.日々の生活に欠かせないもの.確かに間違ってはいない.空気と水がなければ僕たちは行きていけないし,ガスがなければ料理ができず,ガソリンがなければ車に乗れない.しかしこれ以上にこれらに共通することがある.「流体」であるということだ.

「流れる体」と書いて「流体」だが,それは一体何を示すのだろうか.答えは「個体以外のもの全て」だ.実際にはもっと厳格な定義があるが,それはここでは触れないことにしよう.

本稿の目的は,流体を表現する様々な式に触れながら,流体運動を記述するナビエストークス方程式を導出することだ.この方程式は現在様々な分野で活躍する数値流体力学の基礎となる数式で,この方程式さえわかれば流体運動がわかると言っても過言ではない.では,流体を支配する世界を紐解いていこう.

\section{連続の式}
これから登場する様々な式の中で,流体に対して最も強い拘束条件を与えるのが連続の式である.この式は「流体は突然動き出さない」ということを意味する.早速この数式を見てみよう.
\begin{equation}
  \pfrac{\rho}{t}+\pfrac{(\rho{}u)}{x}+\pfrac{(\rho{}v)}{y}+\pfrac{(\rho{}w)}{z}=0
\end{equation}
これをベクトル演算子を使って書いてみると,
\begin{equation}
  \pfrac{\rho}{t}+\mathrm{div}(\rho\bm{V})
\end{equation}
となる.ではどのようにこの数式が導出されるのかを次に述べる.
\subsection{連続の式の導出: 直交座標系の場合}\label{chokou}
とても小さな直方体を考えよう.一辺の長さが$\mathrm{dx}$,$\mathrm{dy}$,$\mathrm{dz}$の直方体のそれぞれの面に流入,流出する流体の質量を考える.

まず$x$軸に垂直な面に$x$軸方向の流れが流入,流出するとき,
\begin{align}
  &流入:\rho{}u|_{x}dydzdt\label{xin}\\
  &流出:\rho{}u|_{x+dx}dydzdt\label{xout}
\end{align}
となり,式(\ref{xout})に関してテイラー展開,一次までを考えると,
\begin{equation}
  \left[\rho{}u+\pfrac{(\rho{}u)}{x}dx\right]dydzdt\label{xouttaylor}
\end{equation}
となり,ここで流入と流出の差,つまり式(\ref{xin})と式(\ref{xouttaylor})の差を取れば,
\begin{equation}
  -\pfrac{(\rho{}u)}{x}dxdydzdt
\end{equation}
を得る.これを$y$に垂直な面,$z$に垂直な面についてそれぞれ考えると,
\begin{align}
  &-\pfrac{(\rho{}u)}{x}dxdydzdt\label{pupx}\\
  &-\pfrac{(\rho{}v)}{y}dxdydzdt\label{pvpy}\\
  &-\pfrac{(\rho{}w)}{z}dxdydzdt\label{pwpz}
\end{align}
がそれぞれ得られる.

さて,式\ref{pupx}$\sim$\ref{pwpz}の総和はこの直方体内の密度変化と体積の積に等しい.ここで直方体内の質量変化は
\begin{equation}
  \pfrac{\rho}{t}dxdydzdt
\end{equation}
で表される.等式で結ぶと,
\begin{equation}
  \pfrac{\rho}{t}dxdydzdt=-\pfrac{(\rho{}u)}{x}dxdydzdt-\pfrac{(\rho{}v)}{y}dxdydzdt-\pfrac{(\rho{}w)}{z}dxdydzdt
\end{equation}
であるから,これを整理して,
\begin{equation}
  \pfrac{\rho}{t}+\pfrac{(\rho{}u)}{x}+\pfrac{(\rho{}v)}{y}+\pfrac{(\rho{}w)}{z}=0
\end{equation}
が得られた.

\subsection{連続の式の導出: 円筒座標系の場合}
先ほど導出した連続の式は直交座標系におけるものだった.次は円筒座標系$(r, \theta, z)$における連続の式を導出してみよう.この場合もとても小さな領域を考えればよい.

まず,任意の円筒の中心から$r$の位置に図\ref{entou}のような微小な領域を取る.
\begin{figure}
  \centering
  \includegraphics[width=0.7\columnwidth]{entou.pdf}
  \caption{円筒座標系における微小領域}
  \label{entou}
\end{figure}
この領域は\ref{chokou}とは異なり,辺の長さが少しだけ異なっている.が,同じように導出に挑戦してみよう.

まず,$r$方向の流体の速度を$V_r$とする.ここで$r$軸に垂直な面に流入,流出する流体の質量は
\begin{align}
  &流入: \rho{}V_r|_{r}rd\theta{}dzdt\label{rin}\\
  &流出: \rho{}V_r|_{r+dr}(r+dr)d\theta{}dzdt\label{rout}
\end{align}
式(\ref{rout})についてテイラー展開し,第一項まで取れば
\begin{equation}
  \left[\rho{}V_rr+\pfrac{(\rho{}V_rr)}{r}dr\right]d\theta{}dzdt
\end{equation}
式(\ref{rin})との差を取り,
\begin{equation}
  -\pfrac{(\rho{}V_rr)}{r}drd\theta{}dzdt
\end{equation}
同様に$\theta{}$,$z$方向についても計算し,
\begin{align}
  &-\pfrac{(\rho{}V_rr)}{r}drd\theta{}dzdt\\
  &-\pfrac{(\rho{}V_\theta{})}{\theta{}}drd\theta{}dzdt\\
  &-\pfrac{(\rho{}V_z)}{r}rdrd\theta{}dzdt
\end{align}

\end{document}
